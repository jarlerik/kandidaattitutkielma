% --- Template for thesis / report with tktltiki2 class ---

\documentclass[finnish]{tktltiki2}

% tktltiki2 automatically loads babel, so you can simply
% give the language parameter (e.g. finnish, swedish, english, british) as
% a parameter for the class: \documentclass[finnish]{tktltiki2}.
% The information on title and abstract is generated automatically depending on
% the language, see below if you need to change any of these manually.
% 
% Class options:
% - grading                 -- Print labels for grading information on the front page.
% - disablelastpagecounter  -- Disables the automatic generation of page number information
%                              in the abstract. See also \numberofpagesinformation{} command below.
%
% The class also respects the following options of article class:
%   10pt, 11pt, 12pt, final, draft, oneside, twoside,
%   openright, openany, onecolumn, twocolumn, leqno, fleqn
%
% The default font size is 11pt. The paper size used is A4, other sizes are not supported.
%
% rubber: module pdftex

% --- General packages ---

\usepackage[utf8]{inputenc}
\usepackage{lmodern}
\usepackage{microtype}
\usepackage{amsfonts,amsmath,amssymb,amsthm,booktabs,color,enumitem,graphicx}
\usepackage[pdftex,hidelinks]{hyperref}

% Automatically set the PDF metadata fields
\makeatletter
\AtBeginDocument{\hypersetup{pdftitle = {\@title}, pdfauthor = {\@author}}}
\makeatother

% --- Language-related settings ---
%
% these should be modified according to your language

% babelbib for non-english bibliography using bibtex
\usepackage[fixlanguage]{babelbib}
\selectbiblanguage{finnish}

% add bibliography to the table of contents
\usepackage[nottoc,numbib]{tocbibind}
% tocbibind renames the bibliography, use the following to change it back
\settocbibname{Lähteet}

% --- Theorem environment definitions ---

\newtheorem{lau}{Lause}
\newtheorem{lem}[lau]{Lemma}
\newtheorem{kor}[lau]{Korollaari}

\theoremstyle{definition}
\newtheorem{maar}[lau]{Määritelmä}
\newtheorem{ong}{Ongelma}
\newtheorem{alg}[lau]{Algoritmi}
\newtheorem{esim}[lau]{Esimerkki}

\theoremstyle{remark}
\newtheorem*{huom}{Huomautus}


% --- tktltiki2 options ---
%
% The following commands define the information used to generate title and
% abstract pages. The following entries should be always specified:

\title{Ohjelmistotuotantomenetelmät}
\author{Jarl-Erik Malmström}
\date{\today}
\level{Aine}
\abstract{Tiivistelmä.}

% The following can be used to specify keywords and classification of the paper:

\keywords{agile, ketterä, open source, avoin}
\classification{} % classification according to ACM Computing Classification System (http://www.acm.org/about/class/)
                  % This is probably mostly relevant for computer scientists

% If the automatic page number counting is not working as desired in your case,
% uncomment the following to manually set the number of pages displayed in the abstract page:
%
% \numberofpagesinformation{16 sivua + 10 sivua liitteissä}
%
% If you are not a computer scientist, you will want to uncomment the following by hand and specify
% your department, faculty and subject by hand:
%
% \faculty{Matemaattis-luonnontieteellinen}
% \department{Tietojenkäsittelytieteen laitos}
% \subject{Tietojenkäsittelytiede}
%
% If you are not from the University of Helsinki, then you will most likely want to set these also:
%
% \university{Helsingin Yliopisto}
% \universitylong{HELSINGIN YLIOPISTO --- HELSINGFORS UNIVERSITET --- UNIVERSITY OF HELSINKI} % displayed on the top of the abstract page
% \city{Helsinki}
%


\begin{document}

% --- Front matter ---

\maketitle        % title page
\makeabstract     % abstract page

\tableofcontents  % table of contents
\newpage          % clear page after the table of contents


% --- Main matter ---

\section{Johdanto}

Tämän kirjoitelman tarkoituksena on tarkastella ohjelmistotuotannon menetelmiä, niiden historiaa, lähestymistapaa ohjelmistotuotantoon ja menetelmien heikkouksia sekä vahvuuksia. Ohjelmistotuotantomentelmät ovat muodostuneet eri aikoina erilaisista lähtökohdista ja eri menetelmät sopivat erilaisiin ohjelmistotuotantoprojekteihin.

Kirjoitelma käy lyhyesti läpi erilaisia lähestymistapoja ohjelmistotuotantoon ja miten erilaisten ohjelmistotuotantoprojektien erityispiirteet vaikuttavat menetelmän valintaan.

Ohjelmistotuotannossa on kaksi perustavanlaatuista vaihetta: analysointivaihe ja rakennusvaihe. Nämä kaksi vaihetta riittävät ohjelmiston totetuttamiseen, jos ohjelmisto on pieni ja tuotettavan ohjelmiston käyttäjät ovat itse toteuttajia. 

Tällaisesta ohjelmistokehityksestä myös asiakkaat ovat valmiita maksamaan, sillä vaiheet pitävät sisällään aidosti luovaa työtä, joka suoraan edistää tuotettavan ohjelmiston käytettävyyttä.

Suuremman ohjelmistotuotantoprojektin täytäntöönpano vaatii lisäksi muita vaiheita, jotka eivät suoraan edistä tuotettavaa ohjelmistoa ja lisäksi kasvattavat ohjelmistotuotannon kustannuksia.\cite{Roy70}

Ohjelmistotuotannon alkuaikoina käytetty ''ohjelmoi ja korjaa'' -mallin sisältää kaksi vaihetta. Ohjelmoidaan ensin ja mietitään vaatimuksia, rakennetta sekä testausta myöhemmin. Mallilla oli useita heikkouksia. Usean korjausvaiheen jälkeen ohjelmakoodi oli niin vaikeasti rakennettu, että oli hyvin kallista muuttaa koodia. Tämä korosti tarvetta suunnitteluaiheelle ennen ohjelmointia.

Usein hyvin suunniteltu ohjelmisto ei vastannut käyttäjien toiveita. Joten syntyi tarve vaatimusmäärittelylle ennen suunnitteluvaihetta. 

Ohjelmistot olivat usein kalliita korjata koska muutoksiin ja testaamiseen oli valmistauduttu huonosti. Tämä osoitti tarpeen eri vaiheiden tunnistamiselle, sekä tarpeen huomioida testaus ja ohjelmiston muuttuminen jo hyvin varhaisessa vaiheessa.\cite{BOE88} 

   

\section{Ohjelmistotuotantomenetelmät}

\subsection{Vesiputousmalli}

\subsubsection{Vaiheet}
1970-luvulla vesiputousmalli vaikutti suuresti eri vaiheisiin perustuviin ohjelmistotuotannon malleihin. Vesiputousmallin lähestymistapa auttoi poistamaan monia aiemmin ohjelmistotuotantoa vaivanneita ongelmia. Vesiputousmallista tuli perusta monille teollisuuden ja hallituksen ohjelmistohankintojen standardeille. \cite{BOE88} Tarkemmin Winston W. Roycen malli sisältää seuraavat vaiheet: järjestelmä- ja ohjelmistovaatimusmäärittely, analyysi, ohjelmistonrakenteen suunnittelu, rakennus, testaus ja ohjelmiston käyttäminen. Perättäisten ohjelmistotuotantovaiheiden välillä on iteraatiota järjestelmän rakenteen tarkentuessa yksityiskohtaisemmaksi tuotannon edetessä. Iteraatioiden tarkoituksena on suunnitelman edetessä pitää muutosvauhti käsiteltävän kokoisena.\cite{Roy70}

\subsubsection{Ohjelmiston suunnittelu}

Lineearinen ohjelmistotuotantoprosessi sisältää huomattavan riskin. Vasta testivaiheessa, menetelmän loppupuolella, saattaa tulla esille ilmiöitä, joita ei ollut mahdollista tarkalleen analysoida aikaisemmassa vaiheessa. Ellei pieni muutos koodissa korjaa ohjelmistoa vastaamaan oletettua käytöstä, vaadittavat muutokset ohjelmiston rakenteeseen saattavat olla niin häiritseviä, että muutokset rikkovat ohjelmistolle asetettuja vaatimuksia. Tällöin joko vaatimuksia tai suunnitelmaa on muutettava. Tässä tapauksessa tuotantoprosessi on palannut alkuun ja kustannusten voidaan olettaa nousevan jopa 100\%.\cite{Roy70}

Ongelman korjaamiseksi vaatimusmäärittelyn jälkeen - ennen analyysia - on tehtävä alustava rakenteen suunnittelu. Näin ohjelmistosuunnittelija välttää talletamiseen tai aika -ja tilavaatimuuksiin liittyvät virheet. Analyysin edetessä ohjelmistosuunnittelijan on välitettävä aika- ja tilavaatimukset sekä operatiivisiset rajoitteet analyysin tekijälle. 

Näin voidaan tunnistaa projektille varatut alimitoitetut kokonaisresurssit tai virheellinen operatiivinen suunnitelma aikaisessa vaiheessa. Vaatimukset ja alustava suunnitelma voidaan iteroida ennen lopullista suunnitelmaa, ohjelmointia ja testausvaihetta.\cite{Roy70}

\subsubsection{Dokumentointi}

On laadittava ymmärrettävä, valaiseva ja ajantasainen dokumentti, jonka jokaisen työntekijän on sisäistettävä. Vähintään yhden työntekijällä on oltava syvällinen ymmärrys koko järjestelmästä, mikä on osaltaan saavutettavissa dokumentin laadinnalla. Ohjelmistuotannon hyvin tärkeä sääntö on erittäin kattava dokumentointi. Ohjelmistosuunnittelijoiden on kommunikoitava rajapintojen(interface) suunnittelijoiden, ja johdon kanssa. Dokumentti antaa ymmärrettävän perustan rajapintojen suunnitteluun ja hallinnollisiin ratkaisuihin. Kirjallinen kuvaus pakottaa ohjelmistosuunnittelijan yksiselitteiseen ratkaisuun ja tarjoaa konkreettisen todistuksen työn valmistumisesta.\cite{Roy70}

Hyvän dokumentoinnin todellinen arvo ilmenee tuotannossa myöhemmin testausvaiheessa, ohjelmistoa käytettäessä sekä uudelleen suunniteltaessa. Hyvän dokumentin avulla esimies voi keskittää henkilöstön ohjelmistossa ilmenneisiin virheisiin. Ilman hyvää dokumenttia, ainoastaan ohjelmistovirheen alkuperäinen tekijä kykenee analysoimaan kyseessä olevan virheen.\cite{Roy70}

Dokumentti helpottaa ohjelmiston käyttöönottoa operatiivinen henkilöstön kanssa. Käyttöönotossa ilmenneiden mahdollisten ohjelmistovirheiden korjaamisessa selkeä dokumentti on välttämätön.\cite{Roy70}   

\subsubsection{Toinen versio}

Dokumentoinnin jälkeen toinen ohjelmistoprojektin onnistumiseen vaikuttava tärkein tekijä on sen alkuperäisyys. Jos kyseessä olevaa ohjelmistoa kehitetään ensimmäistä kertaa, on asiakkaalle toimitettava käyttöönotettava versio oltava toinen versio, mikäli kriittiset rakenteelliset ja operatiiviset vaatimukset on huomioitu. 

Lyhyessä ajassa suhteessa varsinaiseen aikatauluun suunnitellaan ja rakennetaan prototyyppiversio ennen varsinaista rakennettavaa ohjelmistoa. Jos suunniteltu aikataulu on 30 kuukautta, niin pilottiversion aikataulu on esimerkiksi 10 kuukautta. Ensimmäinen versio tarjoaa aikaisen vaiheen simulaation varsinaisesta tuotteesta.\cite{Roy70}

\subsubsection{Testaus}

Testaus on projektin resursseja vaativin vaihe. Testausvaiheessa vallitsee suurin riski taloudellisesti ja ajallisesti. Loppuvaiheessa aikataulua on vähän varasuunnitelmia tai vaihtoehtoja. Alustava suunnitelma ennen analysointia ja ohjelmointia sekä prototyypin valmistaminen ovat ratkaisuja ongelmien löytämiseen ja ratkaisemiseen ennen varsinaiseen testivaiheeseen siirtymistä.

Testivaiheen tulee pääasiallisesti suorittaa siihen erikoistunut asiantuntija, joka ei välttämättä osallistunut varsinaiseen ohjelmointiin. Väite, että ohjelmistosuunnittelija on paras henkilö      
testaamaan suunnitelemansa ohjelmiston, koska ymmärtää aihealueen parhaiten, on merkki siitä että dokumentointi ei ole ollut riittävää. 

Useimmat virheet ovat luonteeltaan ilmiselviä, jotka voidaan löytää visuaalisella tarkastelulla. Jokaisen analyysin ja ohjelmakoodin tulee tarkastaa toinen henkilö, joka ei osallistunut varsinaiseen työhön. Jokainen tietokoneohjelman looginen polku on testattava ainakin kerran.\cite{Roy70}
 
\subsubsection{Asiakas}

Jostain syystä ohjelmiston suunnitelmaan ja aiottuun toimintaan sovelletaan laajaa tulkintaa, jopa aikasemman yhteisymmärryksen jälkeen. On tärkeää sitouttaa asiakas formaalilla tavalla mahdollisimman aikaisessa vaiheessa projektia, näin asiakkaan näkemys, harkinta ja sitoumus vahvistaa kehitystyötä.\cite{Roy70} 	  


\subsection{Itratiivinen menetelmä}
Lineaarisesti vaiheesta toiseen etenevän ohjelmistotuotantomalli ei sopinut erityisesti interaktiivisiin loppukäyttäjien sovelluksiin. Suunnitelmaperustaiset standardit pakottivat dokumentoimaan yksityiskohtaisesti heikosti ymmärretyistä käyttöliittymien vaatimuksista. 

Tästä seurasi käyttökelvottoman ohjelmakoodin suunnittelua ja toteutusta. Vesiputousmallin vaiheet olivat tällaisille projekteille selvästi väärässä järjestyksessä. Erityisesti joillekin ohjelmistoille ei ollut tarvetta yksityiskohtaiselle dokumentaatiolle ennen toteutusta.\cite{BOE88}   

\subsection{Ketterät kehitysmenetelmät}

\section{Ohjelmistojen laadun varmistus}

\subsection{Laadun varmistus vesiputousmallissa}

\subsection{Laadun varmistus ketterissä menetelmissä}

\section{Ohjelmistotuotantomenetelmän valinta}

\subsection{Ohjelmistotuotannon riskit}

\subsection{Ennustettavuus}

\subsection{Muuttuva liiketoimintaympäristö}

\subsection{Organisaation koko}

\subsection{Kehittäjien taidot}

\subsection{}

\section{}

\section{}


% --- Back matter ---
%
% bibtex is used to generate the bibliography. The babplain style
% will generate numeric references (e.g. [1]) appropriate for theoretical
% computer science. If you need alphanumeric references (e.g [Tur90]), use
%
% \bibliographystyle{babalpha}
%
% instead.

\bibliographystyle{babplain}
\bibliography{references-fi}


\end{document}
